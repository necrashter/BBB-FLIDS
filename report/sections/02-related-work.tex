
\section{Related Work}

Performance comparison of different FL algorithms are carried out in \cite{FLperf}, and federated averaging attained the highest accuracy.
However, it has also been shown that FL suffers when the data is non-IID (non-Independent and identically Distributed).
To overcome this, \cite{FL-HC} proposes a hierarchical clustering idea.

A fairly comprehensive literature survey about FL for intrusion detection systems (IDS) is provided in \cite{FLforIDSsurvey}.
A more general survey is given in \cite{IDSusingMLsurvey}, concerning IDS using machine learning.
These papers also provide lots of background information on IDS, which I will omit for the sake of brevity.

When the training data is scarce, anomaly detection systems deliver lots of false positives and exhibit bad performance overall.
Federated learning enables us to leverage the private data distributed in numerous devices without sharing the data directly.
For example, in the collaborative IDS presented in \cite{ColabIDS}, FL is employed in combination with a semi-supervised learning algorithm to mitigate the data scarcity issues.
Similarly, in \cite{FLwirelessIDS}, the privacy-preserving properties of FL are exploited in order to preserve sensitive data during model training.

Furthermore, due to its simplicity, federated learning offers an elegant and flexible framework for building unique machine learning applications.
A single deep neural network is trained for multiple IDS tasks in \cite{mt-dnn-fl} using FL, which even surpassed some models dedicated to a single task.
An example of using blockchain in combination with machine learning is given in \cite{dbf}, which presents a Deep Blockchain Framework for collaborative IDS.
